\documentclass[letterpaper,10pt]{article}
\usepackage{newpxtext}
\usepackage{amsmath, amssymb, amsthm}
\usepackage[dvipsnames]{xcolor}
\usepackage[colorlinks=true,urlcolor=crimson]{hyperref}
\usepackage{titlesec}
\usepackage[margin=.75in]{geometry}
\usepackage{longtable}
\usepackage{etaremune}
\usepackage{enumitem}
\definecolor{crimson}{HTML}{A51C30}


%Setting the font I want:
\renewcommand{\familydefault}{\sfdefault}
\renewcommand{\sfdefault}{ppl}

%Defining the entry command:
\newcommand{\entry}[4]{

\begin{minipage}[t]{.15\textwidth}
\end{minipage}
\hfill\vline\hfill 
\begin{minipage}[t]{0.95\textwidth}
#2 \hfill \textsc{#1}

\textit{#3}

\footnotesize{#4}
\end{minipage}\\\vspace{.25cm}}

\newcommand{\we}[5]{

\begin{minipage}[t]{.15\textwidth}
\end{minipage}
\hfill\vline\hfill 
\begin{minipage}[t]{0.95\textwidth}
#2 \hfill \textbf{#5}

\textit{#3} \hfill \textsc{#1}

\footnotesize{#4}
\end{minipage}\\\vspace{.25cm}}

%Some macros because I'm lazy:
\newcommand{\harv}{Harvard University}
\newcommand{\ufpe}{Federal University of Pernambuco}
\newcommand{\co}{\textcolor{crimson}{*}}

\newcommand{\mycomment}[1]{}

%Section spacing and format:
\titleformat{\section}{\Large\scshape\raggedright}{}{1em}{}[\titlerule]
\titlespacing{\section}{0pt}{3pt}{3pt}
\titleformat{\subsection}{\large\scshape\raggedright}{}{1em}{}
\titlespacing{\subsection}{0pt}{3pt}{3pt}

\begin{document}

\pagenumbering{gobble}
\par{\centering{\Huge Matheus Sobreira Farias}\par}
\par{\centering{\textit{Room 3.410, Science \& Engineering Complex, 150 Western Ave. Allston, MA 02134}}\par}
\par{\centering {\href{mailto:matheusfarias@g.harvard.edu}{matheusfarias@g.harvard.edu} | \href{www.matheussfarias.com}{matheussfarias.com}}\par}
\hfill \textit{Last update: \today}
\vspace{-.25cm}

\section{Education}
\vspace*{.1cm}
\we{2021--2027 (Expected)}{\textbf{\harv}}{Ph.D. in Electrical Engineering}{Hardware-software co-design of efficient hardware architectures for deep learning. Advised by Prof.\ H.\ T.\ Kung (\href{https://www.eecs.harvard.edu/htk/}{link}). GPA: 3.90/4.00}{Cambridge, MA}

\we{2021--2024}{\textbf{\harv}}{M.Sc. in Electrical Engineering}{Relevant Coursework: Hardware Architectures for Deep Learning (A), Tiny Machine Learning (A), High Performance Computing for Science and Engineering (A), Advanced Computer Architecture (A). GPA: 3.92/4.00}{Cambridge, MA}

\we{2016--2021}{\textbf{\ufpe}}{B.Sc. in Electronics Engineering}{
	1st out of 40 students, GPA 8.90/10. Senior Thesis: \textit{iOwlT: Sound Geolocalization System} (\href{https://www.matheussfarias.com/assets/files/SeniorThesis.pdf}{link}).
	}{Recife, Brazil}

\vspace*{-.25cm}
\section{Research}

Detailed information can be found \href{https://www.matheussfarias.com/}{here}.

\vspace{.1cm}

\entry{2021--Present}{\textbf{EfficientAI/TinyML (Meta AI/AFRL collab)}}{\harv}{
	Designing algorithms to improve deep neural networks efficiency (i.e. quantization, pruning, knowledge distillation, etc). Past work adresses bottlenecks such as data conversions, nonidealities, programming time and weight mapping of compute-in-memory crossbars.
}

\entry{2019--2020}{\textbf{iOwlT: Sound Geolocalization System} (\href{https://www.matheussfarias.com/iowlt.html}{link})}{\ufpe}{
	Developed a system using neural networks, adaptive filtering and real-time processing in FPGAs to recognize sound events and determine gun shooters location on a mobile application. Earned 3 international awards at InnovateFPGA 2019 in China (Top 0.7\%).
}

\entry{2019--2020}{\textbf{Lock-in: Nano-Volt Signal Amplifier} (\href{https://www.matheussfarias.com/lockin.html}{link})}{\ufpe}{
	Design and optimization of a phase-sensitive lock-in amplifier advised by the former Minister of Science and Technology of Brazil Prof.\ Sergio Rezende to investigate magnetic properties of IrMn/Py thin films using MOKE technique.
}

\entry{2017}{\textbf{iTraffic: Smart Semaphore Network} (\href{https://www.matheussfarias.com/itraffic.html}{link})}{\ufpe}{
	Design and proposal of an internet of things intelligent system to dynamically choose traffic lights timing to optimize vehicle flow on urban roads using genetic algorithm. Achieved 130\% improvement in the average speed of cars in tested tracks.
}

\entry{2017}{\textbf{Maracatronics: Robotics Team} (\href{https://www.matheussfarias.com/maracatronics.html}{link})}{\ufpe}{
	Member of the collective autonomous soccer sub-team, acting on robots control on Tiva-C microcontroller, computer vision mapping and tracking, and intelligent robots decision-making strategies. Achieved 5th Place at XVI Latin American Robotics Competition.
}

\vspace*{-.25cm}

\section{Publications}
\co denotes equal contribution
\vspace*{.1cm}
%\subsection{Journal}
\begin{etaremune}
	\renewcommand{\labelenumi}{[\theenumi]}
    \item \textbf{M. Farias}, H. T. Kung, ``Breaking Sneak Paths: An Accuracy-Aware Bit Flipping Heuristic for Computing-in-Memory Crossbars'', \textit{in submission}.
	\item \textbf{M. Farias}, ``Semi-Nonnegative Matrix Factorization Improves Compute-in-Memory Crossbars Energy Efficiency'', \textit{in submission}.
	\item \textbf{M. Farias}, W. Martins, H. T. Kung, ``MDM: Manhattan Distance Mapping of DNN Weights for Parasitic-Resistance‑Resilient Memristive Crossbars'', \textit{in submission}, \href{https://arxiv.org/pdf/2511.04798}{https://arxiv.org/pdf/2511.04798}.
	\item \textbf{M. Farias}, H. T. Kung, ``Efficient Reprogramming of Memristive Crossbars for DNNs: Weight Sorting and Bit Stucking'', \textit{ISCAS 2025}, \href{https://arxiv.org/pdf/2410.21730}{https://arxiv.org/pdf/2410.21730}.
    \item \textbf{M. Farias}, H. T. Kung, ``Sorted Weight Sectioning for Energy-Efficient Unstructured Sparse DNNs on Compute-in-Memory Crossbars'', \textit{ISCAS 2025}, \href{https://arxiv.org/pdf/2410.11298}{https://arxiv.org/pdf/2410.11298}.
	\item O. E. Akgun\co, N. Cuevas\co, \textbf{M. Farias}\co, D. Garces\co, ``Tiny Reinforcement Learning for Quadrupled Locomotion Using Decision Transformers'', \href{https://arxiv.org/pdf/2402.13201}{https://arxiv.org/pdf/2402.13201}.
\end{etaremune}
\vspace*{-.25cm}

\section{Conferences}
\vspace*{.1cm}
\begin{etaremune}
	\item \textbf{2025 International Symposium on Circuits and Systems}
	\hfill{London, United Kingdom}
	\item \textbf{2019 International Conference on Field-Programmable Technology}
	\hfill{Tianjin, China}
	\item \textbf{VII Brazilian Symposium on Computing Systems Engineering}
	\hfill{Curitiba, Brazil}
	\end{etaremune}

\vspace*{-.25cm}
\section{Teaching}
\textbf{\harv}
\vspace{-0.5em}
\begin{itemize}[label={}]
    \setlength\itemsep{0.1em}
	\item CS242 -- \textit{Computing at Scale} \hfill\textsc{Fall 2024, Fall 2025 (Head TA)}
	\item CS205 -- \textit{High Performance Computing for Science and Engineering} \hfill\textsc{Spring 2023}
\end{itemize}
\hspace{1.5em}\textbf{\ufpe} 
\vspace{-0.5em}
\begin{itemize}[label={}]
    \setlength\itemsep{0.1em}
    \item ES456 -- \textit{Machine Learning} \hfill\textsc{Fall 2020}
    \item MA326 -- \textit{Complex Variables and Applications} \hfill\textsc{Spring 2018, Fall 2019}
    \item FI007 -- \textit{Physics II: Gravitation, Waves and Thermodynamics} \hfill\textsc{Fall 2017, Spring 2018}
    \item MA026 -- \textit{Calculus I: Limits, Derivatives and Integrals} \hfill\textsc{Fall 2016} 
\end{itemize}
\vspace*{-.25cm}
\mycomment{
\entry{Spring 2023}{CS205 -- \textbf{High Performance Computing} -- Teaching Fellow}{\harv}{
	I led lab activites, held office hours, assisted the teaching staff on grading assignments, and mentored some of the final projects.
	}

\entry{Fall 2020}{ES456 -- \textbf{Machine Learning} -- Teaching Assistant}{\ufpe}{
	I conducted my own activities and lectures off of my own syllabus. Supported the students developing projects and graded work.
	}

\entry{2018--2019}{MA326 -- \textbf{Complex Variables and Applications} -- Teaching Assistant}{\ufpe}{
	I taught once-a-week sessions to support students in their assignments.
	}

\entry{2017--2018}{FI007 -- \textbf{Physics II: Gravitation, Waves and Thermodynamics} -- Teaching Assistant}{\ufpe}{
	I wrote some extra assignments for students interested in Olympic-level Physics, as well as once-a-week sessions to discuss.
	}

\entry{Fall 2016}{MA026 -- \textbf{Calculus I: Limits, Derivatives and Integrals} -- Teaching Assistant}{\ufpe}{
	I taught once-a-week sessions to support students in their assignments.
	}
}

\section{Work Experience}
\vspace*{.1cm}
\we{Summer 2025}{\textbf{Nissan Advanced Technology Center}}{AI Hardware Accelerator Intern}{
	Led AI accelerator architecture exploration and C++ behavioral modeling. Designed vectorized processing elements optimized for self-driving vehicles, synthesizing RTL using Vitis HLS. Conducted architecture performance analysis and benchmarking, delivering reports on resource utilization and timing metrics.
	}{Silicon Valley, CA}
\we{2020--2021}{\textbf{Neurotech}}{Machine Learning Operations Intern}{
	Implemented 5 machine learning algorithms for creditworthiness assessment system. Built end-to-end ML pipeline using PyTorch for model development, ONNX for production deployment, and MLflow for experiment tracking and model management.
	}{Recife, Brazil}
\we{2016--2018}{\textbf{Espaço Diferencial}}{Co-Founder and Teacher}{
	Idealized a non-profit school for underpriviledge students in introductory engineering classes. Managed the action strategy planning that impacted over 200 students with a team of 10 teachers. Taught Physics at the undergraduate level.
	}{Recife, Brazil}
\vspace*{-.25cm}
\section{Awards and Recognitions}
\vspace*{.1cm}

\entry{2025}{\textbf{Full Member at Sigma Xi, the Scientific Research Honor Society}}{International}{
	Nominated to the world’s largest general research honor society. Founded in 1886, with 200+ Nobel laureates among its members.
}

\entry{2025}{\textbf{R\$100k Prize at Who Wants to be a Millionaire}}{Brazil}{
	Correctly answered 11 out of 15 questions in the world's most competitive trivia game.	
}

\entry{2024}{\textbf{1st Ecossis Award of Innovation and Sustainability at Mostratec (The biggest S\&T fair in LatAm)}}{Brazil}{
	SIMBA is an AI-powered sound localization system that monitors \textit{Antilophia bokermanni}, an endangered bird of cultural value in Brazil.	
}

\entry{2024}{\textbf{MIT Innovator Under 35 in Artificial Intelligence}}{Brazil}{
	Title given to top innovators in Science and Technology under the age of 35.	
}

\entry{2024}{\textbf{Líder Estudar Fellow (``the Brazilian Rhodes Scholarship'')} }{Brazil}{
	One of the 26 students over 45,000 candidates -- the most competitive scholarship in the country.	
}

\entry{2023}{\textbf{Bronze Medal at the Online Young Physicists' Tournament}}{Online}{
	8th place at the Online Young Physicists' Tournament 2023.	
}

\entry{2023}{\textbf{Silver Medal at the International Young Physicists' Tournament (Physics World Cup)}}{Murree, Pakistan}{
	2nd place at the 36th International Young Physicists' Tournament 2023 Pakistan.	
}

\entry{2021--Present}{\textbf{Behring Foundation Fellowship}}{Harvard University}{
	Honored by the Behring Foundation with a fellowship to cover my graduate studies at Harvard.	
}

\entry{2019}{\textbf{Three International Awards at InnovateFPGA 2019 Contest}}{Tianjin, China}{
	2 Silver Awards (\textit{Grand Finals} and \textit{Regional Finals}) and Community Award (\textit{Best project in America}). 2nd out of 270 teams with iOwlT.
}

\entry{2019}{\textbf{PIBIC/CNPq funding to do research (``the Brazilian National Science Foundation fellowship'')}}{Brazil}{
	Awarded by national government funding to do research for Lock-in: Nano-Volt Signal Amplifier.	
}

\entry{2017}{\textbf{5th Place at XVI Latin American Robotics Competition}}{Latin America}{
	In the Small Size League category of autonomous soccer with Maracatronics: Robotics Project.
}

\entry{2017}{\textbf{1st Place at Embedded Systems Regional Contest}}{Brazil}{
	1st out of 14 teams with iTraffic: Smart Semaphore Network.
}

\entry{2015}{\textbf{Honorable Mention at Brazilian Physics Olympiad}}{Brazil}{
	One of the 180 medalists over more than 300,000 contestants.
}
\vspace*{-.25cm}
%\section{Other Qualifications}
%\begin{longtable}{rp{12cm}}
%\textsc{Languages:}&\textbf{Native} Portuguese, \textbf{Advanced} English\\
%\textsc{Softwares:}&Tensorflow, scikit, OpenCV, Quartus, Matlab, KiCAD, Origin, Adobe Photoshop, Illustrator, \LaTeX\\	
%\end{longtable}

%\mycomment{
\section{Diversity, Inclusion \& Outreach}
\begin{itemize}
	\item President of Brazil Conference 2025
    \hfill\textsc{2024--2025}

    \item Vice President of Brazil Conference 2024
    \hfill\textsc{2023--2024}

    \item Brazilian Team Leader at the Online Young Physicists' Tournament
    \hfill\textsc{2023}

	\item Brazilian Team Leader at the International Young Physicists' Tournament in Pakistan
    \hfill\textsc{2023}

    \item Author of the Experimental Exam for the Brazilian selective to the International Physics Olympiad
    \hfill\textsc{2023}

    \item Leader of the Diversity \& Inclusion branch at the Harvard Brazilian Association
	\hfill\textsc{2022--2023}

	\item Judge for the InnovateFPGA 2022 Contest
    \hfill\textsc{2022}

    \item Judge for the 4th Brazilian Physicists' Tournament
    \hfill\textsc{2021}

    \item Officer of the School of Engineering at the Harvard Brazilian Association
	\hfill\textsc{2021--2024}

    \item Judge for the International Young Physicists' Tournament Brazil
	\hfill\textsc{2021--Present}
	\end{itemize}
\vspace*{-.25cm}
\section{Talks}
\entry{2025}{\textbf{Who Wants to be a Millionaire} (\href{https://globoplay.globo.com/v/13718386/}{link})}{Trivia + my journey and projects (only available to watch in Brazil)}{}
%}

\entry{2025}{\textbf{Backstage PodCast -- How's the Mind of a Harvard student} (\href{https://www.youtube.com/watch?v=lUqDgKRqxhY}{link})}{My favorite PodCast! Talked about journey and projects}{}
%}

\entry{2025}{\textbf{Conferência Nacional de Defesa e Difusão da Ciência} (\href{https://www.youtube.com/live/LHjNjV-f6q0?si=0vh8dLnf76PwcLZl&t=20177}{link})}{Applied Knowledge: The Academic, Enterpreneur, and Corporative Perspective of Science}{}
%}

\entry{2024}{\textbf{Educar -- Terra} (\href{https://www.terra.com.br/noticias/educacao/da-recuperacao-em-fisica-ao-phd-em-harvard-conheca-a-historia-do-pernambucano-premiado-pelo-mit,3ed69c5460e41efbfe796dffdf6c47c4shi065se.html}{link})}{From failing Physics to a Ph.D at Harvard: discover the MIT award recipient from Pernambuco}{}
%}

\entry{2024}{\textbf{Crusoé -- O Antagonista} (\href{https://crusoe.com.br/edicoes/325/a-historia-do-primeiro-brasileiro-phd-em-engenharia-em-harvard/}{article}) (\href{https://www.youtube.com/watch?v=wmis_W-EFKk}{video})}{The story of the first Brazilian EE Ph.D student at Harvard}{}
%}

\entry{2024}{\textbf{Mais Você -- Globo (biggest Brazilian TV channel)} (\href{https://globoplay.globo.com/v/12673677/}{link})}{An interview about my journey and projects}{}
%}

\entry{2023}{\textbf{Futuras Cientistas -- Ministry of Science, Technology \& Innovation of Brazil} (\href{https://www.youtube.com/watch?v=g5Inc6qXO6k&ab_channel=FuturasCientistas}{link})}{Technology and its Social Impact}{}
%}

\entry{2021}{\textbf{PodCast Ph.D nos EUA} (\href{https://open.spotify.com/episode/37Rn34gdy7AxNgHeAvdIH0?si=11bf338624d04113}{part 1}) (\href{https://open.spotify.com/episode/7zYdb51lXeKbfpULOgpQlx?si=d1dbd10b9197486e}{part 2})}{Journey to become a Ph.D student}{}
\end{document}